%\special{pdf:encrypt ownerpw (wjw) userpw (201124) length 128 perm 2052}

\documentclass[answers]{exam}
\usepackage[UTF8]{ctex} % 添加ctex宏包以支持中文
\usepackage{amsmath}
\usepackage{amsthm}
\usepackage{amsfonts}
\usepackage{amssymb}
\usepackage{mathrsfs}
\usepackage{graphicx}
%\usepackage{graphicx}
%\usepackage{float}
\renewcommand{\qedsymbol}{$\blacksquare$}
%\usepackage{background}
%\usepackage{xcolor}
\usepackage[hidelinks]{hyperref}
% 定义水印
%\backgroundsetup{
%  scale=1,  % 水印的大小
%  color=red!50,  % 水印的颜色和透明度
%  angle=0,  % 水印的角度
%  position=current page.north east,  % 水印的位置(右上角)
%  vshift=-1cm,  % 垂直偏移量
%  hshift=-3cm,  % 水平偏移量
%  contents={微信公众号:小温Learning},  % 水印内容
%}
\usepackage{fancyhdr}
%\usepackage{geometry}%设置页边距宏包
%\geometry{a4paper,left=2.5cm,right=2.5cm,top=2cm,bottom=2.5cm}%页边距的具体设置

%\usepackage[a4paper,top=2.5cm,bottom=2.5cm,left=3cm,right=3cm,% margins
%			headheight=1.5cm,headsep=1.5em,
%			footskip=2em,
%			]{geometry}
% Set up fancy header/footer
\pagestyle{fancy}
%\fancyhead[LO,L]{1}
%\fancyhead[CO,C]{1}
%\fancyhead[RO,R]{\today}
%\fancyfoot[LO,L]{xiaowen}
%\fancyfoot[CO,C]{\thepage}
%\fancyfoot[RO,R]{yibao}
%\renewcommand{\headrulewidth}{0.4pt}
%\renewcommand{\footrulewidth}{0.4pt}