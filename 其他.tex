\section{其他题}
\begin{questions}
\question{
长为 $\pi$ 的两端固定的弦自由振动,如果初始位移为 $ x \sin 2 x $, 初始速度为 $ \cos 2 x $ ,则其定解条件是
}
\begin{solution}
    问题描述:我们有一根长度为 $\pi$ 的弦,两端固定,它在时间 $t$ 和位置 $x$ 上的振动用函数 $u(x, t)$ 来表示。这个振动满足波动方程 $u_{tt} = a^2 u_{xx}$,其中 $a$ 是一个正常数。在 $t = 0$ 时刻,弦的初始位移是 $x\sin(2x)$,初始速度是 $\cos(2x)$。

定解条件:定解条件规定了问题的边界和初始状态:\\
    弦的两端在任何时刻都保持固定:$u(0, t) = u(\pi, t) = 0$,这表示弦的两端被固定住.\\
    初始时刻 $t = 0$ 时,弦的位移是 $x\sin(2x)$:$u(x, 0) = x\sin(2x)$.\\
    初始时刻 $t = 0$ 时,弦的速度是 $\cos(2x)$:$u_t(x, 0) = \cos(2x)$.

    $$\left\{\begin{array}{l}u_{t t}=a^{2} u_{x x} \quad 0< x< \pi, t>0 . \\ u(x, 0)=x \sin 2 x, u_{t}(x, 0)=\cos 2 x \\ u(0, t)=u(\pi, t)=0 .\end{array}\right. $$
\end{solution}

\question{
已知边值问题 $\left\{\begin{array}{l}X^{\prime \prime}(x)+\lambda X(x)=0 \\ X^{\prime}(0)=X(\pi)=0,\end{array}\right.$
则其固有函数 $ X_{k}(x)= $
}
    \begin{solution}
        分三种情况讨论:

$(1)$ 当 $ \lambda<0 $ 时, $$X(x)=c_{1} e^{\sqrt{-\lambda} x}+c_{2} e^{-\sqrt{-\lambda} \cdot x} $$

代入边值条件有 
$$\left\{\begin{array}{ll}c_{1} e^{\sqrt{-\lambda} \pi}+c_{2} e^{-\sqrt{\lambda} \pi}=0 . &  \\ c_{1} \sqrt{-\lambda}-c_{2} \sqrt{-\lambda}=0 .\end{array}\right. $$
$\Rightarrow c_{1}=c_{2}=0 .$从而 $ X(x) \equiv 0. $

$(2)$ 当 $ \lambda=0 $ 时, $ X(x)=c_{1} x+c_{2} $

代入边值条件 
$$ \left\{\begin{array}{l}c^{\pi+} c_{2}=0 \\ c_{1}=0 .\end{array} \Rightarrow c_{1}=c_{2}=0\right. $$ 
从而 $ X(x) \equiv 0 $

$(3)$ 当 $ \lambda>0 $ 时 $ X(x)=c_{1} \cos \sqrt{\lambda} x+c_{2} \sin \sqrt{\lambda} x $

代入边值条件 
$$ \left\{\begin{array}{l}c_{1} \cos \sqrt{\lambda} \pi+c_{2} \sin \sqrt{\lambda} \pi=0 \\ c_{2} \sqrt{\lambda}=0 .\end{array}\right. $$
$$\Rightarrow \lambda=\lambda_{k}=\left(k+\frac{1}{2}\right)^{2} ,k=0,1,2 \ldots$$
因此 $ X_{k}(x)=C_{k} \cos \left(k+\frac{1}{2}\right) x \quad k=0,1,2 $
    \end{solution}

    \question{
    求问题 $\left\{\begin{array}{l}\frac{\partial^{2} u}{\partial t^{2}}=a^{2} \frac{\partial^{2} u}{\partial x^{2}} \\ u(x, 0)=\sin 2 x\\\left.\frac{\partial u}{\partial t}\right|_{t=0}=3 x^{2}\end{array}\right.$ 的解.
    }
    \begin{solution}
        对于初值问题 
$$ \left\{\begin{array}{l}\frac{\partial^{2} u}{\partial t^{2}}-a^{2} \frac{a^{2} u}{\partial x^{2}}=0 \\ u(x, 0)=\varphi(x), u_{t}(x, 0)=\psi(x)\end{array}\right. $$
根据达朗贝尔公式,其解为 
$$ u(x, t)=\frac{1}{2}[\varphi(x-a t)+\varphi(x+a t)]+\frac{1}{2 a} \int_{x-a t}^{x+a t} \psi(\xi) d \xi $$
代入公式之中,因此上述问题的解为:
$$\begin{aligned}
u(x, t)&=\frac{1}{2}[\sin (2 x-2 a t)+\sin (2 x+2 a t)]  +\frac{1}{2 a} \int_{x-a t}^{x+a t} 3 \xi^{2} d \xi\\&=\frac{1}{2}[\sin (2 x-2 a t)+\sin (2 x+2 a t)]+a^{2} t^{3}+3 x^{2} t \end{aligned}$$

如果公式不记得可以自行推导:(下面我来推导一遍)

一维齐次波动方程的通解是 $ u(x, t)=F(x-a t)+G(x+a t) $
由初始条件 $ t=0 $ 时 $ u=\sin 2 x, \frac{\partial u}{\partial t}=3 x^{2} $
则有 
$$ \left\{\begin{array}{l}F(x)+G(x)=\sin 2 x \\ -a F^{\prime}(x)+a G^{\prime}(x)=3 x^{2}\end{array}\right. $$
于是 $ -F^{\prime}(x)+G^{\prime}(x)=\frac{3 x^{2}}{a} $ 两边积分得 $ -F(x)+G(x)=\frac{x^{3}}{a}+c $
$$
\begin{aligned}
\Rightarrow \quad F(x) & =\frac{1}{2} \sin 2 x-\frac{1}{2 a} \cdot x^{3}-\frac{c}{2} \\
G(x) & =\frac{1}{2} \sin 2 x+\frac{1}{2 a} \cdot x^{3}+\frac{c}{2}
\end{aligned}
$$
因此
$$
\begin{aligned}
u(x, t) & =F(x-a t)+G(x+a t) \\
& =\frac{1}{2} \sin (2 x-2 a t)-\frac{1}{2 a}(x-a t)^{3}-\frac{c}{2} \\
& +\frac{1}{2} \sin (2 x+2 a t)+\frac{1}{2 a}(x+a t)^{3}+\frac{c}{2} \\
& =\frac{1}{2}[\sin (2 x-2 a t)+\sin (2 x+2 a t)] \\
& +a^{2} t^{3}+3 x^{2} t
\end{aligned}
$$
和上面答案一致.
    \end{solution}

\question{
用分离变量法解下列混合问题
$$
\left\{\begin{array}{l}
\frac{\partial^{2} u}{\partial t^{2}}=a^{2} \frac{\partial^2 u}{\partial x^{2}} \\
u(0, t)=u(\pi, t)=0 \\
u(x, 0)=2 x(\pi-x) \\
u_{t}(x, 0)=3 \sin 2 x
\end{array}\right.
$$
}
    \begin{solution}
        $$
\text { 令 } u(x, t)=X(x) \cdot T(t), \quad \text { 则 }\left\{\begin{array}{l}
X^{\prime \prime}(x)+\lambda X(x)=0 \\
T^{\prime \prime}(t)+\lambda a^{2} T(t)=0
\end{array}\right.
$$
由边界条件 $ u(0, t)=u(\pi, t)=0 $ 知 $ x(0) T(t)=x(\pi) T(t)=0 $ 由于 $ T(t) \neq 0 $. 则 $ x(0)=x(\pi)=0 $. 所以 
$$\left\{\begin{array}{l}x^{\prime \prime}(x)+\lambda x(x)=0 \\ x(0)=x(\pi)=0\end{array}\right. $$ 
经讨论知 $ \lambda>0 $ 时有非零解.
$$
X(x)=c_{1} \cos \sqrt{\lambda} x+c_{2} \sin \sqrt{\lambda} x
$$
代入边界条件有 
$$ \left\{\begin{array}{l}c_{1}=0 \\ c_{1} \cos \sqrt{\lambda} \pi+c_{2} \sin \sqrt{\lambda} \pi=0\end{array} \Rightarrow \lambda=\lambda_{k}^{2}=k^{2}, k=1,2 \cdots\right. $$
$$
\text { 因此 } X(x)=C_{k} \sin k x \quad k=1,2, \cdots
$$
将 $ \lambda_{k} $ 代 $ \lambda T^{\prime \prime}(t)+\lambda a^{2} T(t)=0 $ 中可得其通解为
$$
T_{k}(t)=a_{k} \cos k a t+b_{k} \sin k a t . k=1,2 \cdots
$$
则 
$$ u_{k}(x, t)=X_{k}(x) T_{k}(t)=\left(A_{k} \cos k a t+B_{k} \sin k a t\right) \sin k x $$ 
由叠加原理得 
$$ u(x, t)=\sum_{k=1}^{\infty}\left(A_{k} \cos k a t+B_{k} \sin k a t\right) \sin k x $$
. 由于 
$$ \left\{\begin{array}{l}u(x, 0)=\sum\limits_{k=1}^{\infty} A_{k} \sin k x=2 x(\pi-x) \\ u_{t}(x, 0)=\sum\limits_{k=1}^{\infty} k a B_{k} \sin k x=3 \sin 2 x\end{array}\right. $$

$$
\begin{aligned}
A_{k} & =\frac{2}{\pi} \int_{0}^{\pi} 2 \xi(\pi-\xi) \sin k \xi d \xi \\
& =\frac{4}{k \pi} \int_{0}^{\pi}\left(\xi^{2}-\pi \xi\right) \cdot d(\cos k \xi) \\
& =\frac{4}{k \pi}\left[\left.\left(\xi^{2}-\pi \xi\right) \cos k \xi\right|_{0} ^{\pi}-\int_{0}^{\pi}(2 \xi-\pi) \cdot \cos k \xi d \xi\right] \\
& =\frac{4}{k \pi}\left[-\frac{1}{k} \int_{0}^{\pi}(2 \xi-\pi) d(\sin k \xi)\right] \\
& =-\frac{4}{k^{2} \pi}\left[\left.(2 \xi-\pi) \sin k \xi\right|_{0} ^{\pi}-2 \int_{0}^{\pi} \sin k \xi d \xi\right] \\
& =\frac{8}{k^{2} \pi} \int_{0}^{\pi} \sin k \xi d \xi=\frac{8}{k^{3} \pi}\left(-\left.\cos k \xi\right|_{0} ^{\pi}\right) \\
& =\frac{8}{k^{3} \pi}\left[1-(-1)^{k}\right]
\end{aligned}
$$
$$
B_{k}=\frac{2}{k \pi a} \int_{0}^{\pi} 3 \sin 2 \xi \sin k \xi d \xi=\left\{\begin{array}{ll}
0 & k \neq 2 \\
\frac{3}{2 a}, & k=2
\end{array}\right.
$$
因此
$$ u(x, t)=\frac{3}{2 a} \sin 2 a t \sin 2 x+\sum_{k=1}^{\infty} \frac{8\left[1-(-1)^{k}\right]}{k^{3} \pi} \cos k at\cdot\sin k x $$
    \end{solution}
\end{questions}




