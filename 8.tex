\begin{questions}
\question{ 证明方程 $ \frac{\partial u}{\partial t}=a^{2} \frac{\partial^{2} u}{\partial x^{2}}+c u(c \geqslant 0) $ 具狄利克雷边界条件的初边值问题解的唯一性与稳定性。
}

\begin{solution}
    首先,考虑方程 $ \frac{\partial u}{\partial t} = a^{2} \frac{\partial^{2} u}{\partial x^{2}} + cu $,通过变换 $ v(x, t) = u(x, t)e^{-ct} $,我们得到新的方程:
$$
\frac{\partial v}{\partial t} = a^{2} \frac{\partial^{2} v}{\partial x^{2}}
$$

现在,我们有一个标准的热传导方程。我们可以使用热传导方程的极值原理来证明.

不妨考虑矩形区域 
$Q=\left\{(x, t) \mid 0< x < L , 0 < t \leqslant T\right\}$. 作变换后 $v(x, t)$ 满足如下初边值问题: 
$$\left\{\begin{array}{l}v_{t}=a^{2} v_{x x} \\ v(x, 0)=f(x) \\ v(0, t)=\varphi(x), v(L, t)=\psi(x)\end{array}\right.$$
先证解的唯一性

设 $ v(x, t) $ 和 $ v_{2}(x, t) $ 是在函数空间 $ C^{2,1}(Q) \bigcap c(\overline{Q}) $ 中的两个解. 记 $ v(x, t)=v_{1}(x, t)-v_{2}(x, t) $, 那么$ v(x, t) $满足齐次初边值条件的定解问题:
$$\left\{\begin{array}{ll}
v_{t}=a^{2} v_{x x} ,\quad 0 < x < L, 0 < t < T \\
v(x, 0)=0 ,\quad 0 \leqslant x \leqslant L \\
v(0, t)=v(L, t)=0, \quad 0 \leqslant t \leqslant T
\end{array}\right.
$$

设 $\Gamma$ 为 $Q$ 的抛物边界, 由极值定理,对 $\forall(x, t) \in \overline{Q}$
成立 
$$0=\min _{\Gamma} v(x, t) \leqslant v(x, t) \leqslant \max_{ \overline Q} v(x, t)=0$$
即在区域 $\overline Q$ 上, $ v(x, t)=0 $.
则 $v_{1}(x, t)=v_{2}(x, t) $ ,于是 $ v(x, t) $ 唯一,也即 $ u(x, t) $ 唯一.

下面考虑解的稳定性

设 $ v_{i}(x, t) \quad(i=1,2) $ 是如下问题的解
$$ \left\{\begin{array}{l}v_{t}=a^{2} v_{x x} \\ v(x, 0)=f_{i}(x) \\ v(0, t)=\varphi_{i}(t), v(L, t)=\psi_{i}(t)\end{array}\right. $$

当 $ 0 \leqslant x \leqslant L , 0 \leqslant t \leqslant T $ 时
如果有 $|f(x)| < \varepsilon,|\varphi(t)| < \varepsilon,|\psi(t)| < \varepsilon$
那么由极值原理得 $ |v(x, t)|<\varepsilon $,
则 $|u(x, t)|=\left|v(x, t) \cdot e^{c t}\right|<\varepsilon e^{c T}$
即 $u(x, t)$ 是稳定的.

~\\
或者按照如下证明(方法二):

在矩形域 $Q=\left\{\alpha < x < \beta , 0 < t \leqslant T\right\}$ 上考虑如下第一边值问题。
$$
\begin{array}{l}
\text { (1) }\left\{\begin{array}{l}
u_{t}=a^{2} u_{x x}+c u, c \geqslant 0,(x, t) \in Q \\
\left.u\right|_{x=\alpha}=f_{1}(t),\left.u\right|_{x=\beta}=f_{2}(t), 0 \leqslant t \leqslant T . \\
\left.u\right|_{t=0}=\varphi(x), \alpha \leqslant x \leqslant \beta
\end{array}\right. \\
\text { 作变换 } v(x, t)=u(x, t) \cdot e^{-c t} \cdot \text { 则 } v(x, t) \text { 满足 } \\
\left\{\begin{array}{l}
v_{t}=a^{2} v_{x x} \\
v(\alpha, t)=e^{-c t} \cdot f_{1}(t), v(\beta, t)=e^{-c t} \cdot f_{2}(t) \\
\left.v\right|_{t=0}=\varphi(x)
\end{array}\right. 
\end{array}
$$

设
$$B=\max \left\{\sup _{[0, T]}\left|e^{-c t} \cdot f(t)\right|, \sup _{[0, T]}\left|e^{-c t} \cdot f_{2}(t)\right|\right\}$$
$$M=\sup _{[\alpha, \beta]}|\varphi(x)|$$
则由热传导方程的极值定理
有 
$|v(x, t)| \leqslant \max \{B , M\}$
$$
\begin{aligned}
|u(x, t)| & =\left|v(x, t) \cdot e^{c t}\right| \\
& \leqslant \max \left\{M e^{c t}, B e^{c t}\right\}
\end{aligned}
$$
设 $u_{1} , u_{2}$ 是问题 $(1)$ 的两个解,则它们的差 $u(x, t)=u_{1}(x, t)-u_{2}(x, t)$ 满足如下第一边值问题:
$$
\left\{\begin{array}{l}
u_{t}=a^{2} u_{x x}+c u \\
\left.u\right|_{t=0}=0 ,\quad \alpha \leqslant x \leqslant \beta \\
\left.u\right|_{x=\alpha}=\left.u\right|_{x=\beta}=0
\end{array}\right.
$$
此时 $ M=B=0 $ 即有 $ |u(x, t)| \leqslant 0 $
因此 $ u(x, t)=0 . \Rightarrow u_{1}=u_{2} $, 得证唯一性.

设 $u_1$ 满足如下边值问题
$$
\begin{array}{l}
\left\{\begin{array}{l}
u_{t}=a^{2} u_{x x}+c u \\
\left.u\right|_{t=0}=\varphi^{(1)}(x), \alpha \leq x \leq \beta \\
\left.u\right|_{x=\alpha}=f_{1}^{(1)}(t),\left.u\right|_{x=\beta}=f_{2}^{(1)}(t)
\end{array}\right. \\
u_{2} \text { 满足 }\left\{\begin{array}{l}
u_{t}=a^{2} u_{x x}+c u \\
u_{t=0}=\varphi^{(2)}(x) \quad \alpha \leqslant x \leqslant \beta \\
\left.u\right|_{x=\alpha}=f_{1}^{(2)}(t),\left.u\right|_{x=\beta}=f_{2}^{(2)}(t)
\end{array}\right. \\
\end{array}
$$
则 $u=u_{1}-u_{2}$ 满足如下边值问题
$$
\left\{\begin{array}{l}
u_{t}=a^{2} u_{x x}+c u \\
\left.u\right|_{t=0}=\varphi^{(1)}(x)-\varphi^{(2)}(x) \\
\left.u\right|_{x=\alpha}=f_{1}^{(1)}(t)-f_{1}^{(2)}(t),\left.u\right|_{x=\beta}=f_{2}^{(1)}(t)-f_{2}^{(2)}(t)
\end{array}\right.
$$
$\text { 对 } \forall \varepsilon>0 , \exists \delta>0 \text { 时 }$
$$
\begin{aligned}
\text { 使得当 } \max _{[\alpha, \beta]}\left|\varphi^{(1)}(x)-\varphi^{(2)}(x)\right| & <\delta \\
\max _{[0, T]}\left|f_{1}^{(1)}(t)-f_{1}^{(2)}(t)\right| & <\delta \\
\max _{[0, T]}\left|f_{2}^{(1)}(t)-f_{2}^{(2)}(t)\right| & <\delta \text { 成立时 } \\
\max |u(x, t)|=\max \left|u_{1}-u_{2}\right| & \leqslant \delta \cdot \max \left\{M e^{c t}, B e^{c t}\right\} \\
& \leqslant \delta \cdot \max \left\{M e^{c T}, B e^{c T}\right\} \\
\end{aligned}
$$
 取 
$\delta  =\frac{\varepsilon}{\max \left\{M e^{c T}, B e^{c T}\right\}}$,
则 $ \max \left|u_{1}(x, t)-u_{2}(x, t)\right|<\varepsilon $.
因此该初边值问题的解为稳定的。
\end{solution}

\question{ 利用证明热传导方程极值原理的方法. 证明满足方程 $ \frac{\partial^{2} u}{\partial x^{2}}+\frac{\partial^{2} u}{\partial y^{2}}=0 $ 的函数在有界闭区域上 的最大值不会超过它在边界上的最大值。
}

\begin{solution}
    设 $\Omega$ 是 $ R^{2} $ 中的有界闭区域,边界 $ \partial \Omega $ 光滑 且 $ u \in C^{2}(\Omega) \cap C(\overline{\Omega}), \Delta u=u_{x x}+u_{y y}=0$,$(x, y) \in \Omega $

用反证法. 记 $m$ 是 $u$ 在边界 $ \partial \Omega $ 上的最大值 设 $u$ 在闭区域 $ \overline{\Omega} $ 上的最大值 $ M $ 在 $ \Omega $ 的内部点 $ M_{0}\left(x_{0}, y_{0}\right) $ 取得.  设 $ u\left(x_{0}, y_{0}\right)=M $, 显然 $ M \geqslant m $

若 $M > m$, 我们作一辅助函数
$$
v(x, y)=u(x, y)+\frac{M-m}{4 R^{2}}\left[\left(x-x_{0}\right)^{2}+\left(y-y_{0}\right)^{2}\right]
$$
其中 $ R $ 是闭区域 $\overline{\Omega}$ 的半径.

注意到 $ v\left(x_{0}, y_{0}\right)=u\left(x_{0}, y_{0}\right)=M $,
且在边界 $ \partial \Omega $ 上有 $ v(x, y) \leqslant m+M-m=M $.
这表明函数 $ v(x, y) $ 在区域 $ \Omega $ 内部的某一点 $ M_{1}\left(x_{1}, y_{1}\right) $ 上取到最大值.
在该点外 
$$ \quad \Delta V=v_{x x}+v_{y y}=u_{x x}+u_{y y}+\frac{M-m}{R^{2}}=\frac{M-m}{R^{2}}>0 $$
但在 $ M_{1} $ 点处. $ v_{x x} \leqslant 0 , v_{y y} \leqslant 0 $, 即 $ \Delta V \leqslant 0 $.
这与上式矛盾,故 $ M=m $. 得证
\end{solution}

\question{ 证明: 当 $ \varphi(x, y) $ 为 $ R^{2} $ 上的有界连续函数, 且 $ \varphi  \in L^{1}\left(R^{2}\right) $ 时, 二维热传导方程柯西问题的解,当 $ t \rightarrow+\infty $ 时,以 $ t^{-1} $ 衰减率趋于零
}

\begin{solution}
    要证明二维热传导方程的柯西问题在 $ t \rightarrow +\infty $ 时以 $ t^{-1} $ 的衰减率趋于零,首先回顾二维热传导方程和其解的表达式。

二维热传导方程的柯西问题如下
$\left\{\begin{array}{l}
u_{t}=a^{2} \Delta u \\
\left.u\right|_{t=0}=\varphi(x, y)
\end{array}\right.$

对方程和初始条件作傅里叶变换
$$
\begin{array}{l}
F[u(x, y, t)]=\tilde{u}\left(\lambda_{1}, \lambda_{2}, t\right)=\iint_{R^{2}} u(x, y, t) e^{-i\left(\lambda_{1} x+\lambda_{2} y\right)} d x d y \\
F[\varphi(x, y)]=\tilde{u}\left(\lambda_{1}, \lambda_{2}\right)=\iint_{R^{2}} \varphi(x, y) e^{-i\left(\lambda_{1} x_{1}+\lambda_{2} x_{2}\right)} d x d y \\
\end{array}
$$
则有
$$
\left\{\begin{array}{l}
\frac{d \tilde{u}}{d t}=\left[\left(i \lambda_{1}\right)^{2}+\left(i \lambda_{2}\right)^{2}\right] a^{2} \tilde{u}^{2}=-\left(\lambda_{1}^{2}+\lambda_{2}\right)^{2} a^{2} \tilde{u} \\
\tilde{u}\left(\lambda_{1}, \lambda_{2}, 0\right)=\tilde{u}\left(\lambda_{1}, \lambda_{2}\right)
\end{array}\right.
$$
可看作 $\tilde u$ 关于 $t$ 的常微分方程的初值问题

解得: 
$$\tilde{u}\left(\lambda_{1}, \lambda_{2}, t\right)=\tilde{u}\left(\lambda_{1}, \lambda_{2}\right) \cdot e^{-a^{2}\left(\lambda_{1}^{2}+\lambda_{1}^{2}\right) t} $$
于是 
$$ u(x, y, t)=F^{-1}\left[\tilde{u}\left(\lambda_{1}, \lambda_{2}, t\right)\right]$$

$$
=F^{-1}\left[\tilde{u}\left(\lambda_{1}, \lambda_{2}\right) \cdot e^{-a^{2}\left(\lambda_{1}^{2}+\lambda_{2}^{2}\right) t}\right]
$$
利用卷积定理
有 
$$u(x, y, t)=F^{-1}\left[\tilde{\varphi}\left(\lambda_{1}, \lambda_{2}\right)\right] * F^{-1}\left[e^{-a^{2}\left(\lambda_{1}^{2}+\lambda_{2}^{2}\right) t}\right] $$
$$
\begin{aligned}
F^{-1}\left[e^{-a^{2}\left(\lambda_{1}^{2}+\lambda_{2}^{2}\right) t}\right] & =\frac{1}{(2 \pi)^{2}} \iint_{R^{2}} e^{-a^{2}\left(\lambda_{1}^{2}+\lambda_{2}^{2}\right) t} \cdot e^{i\left(\lambda_{1} x+\lambda_{2} y\right)} d \lambda_{1} d \lambda_{2} \\
& =\left(\frac{1}{2 \pi} \int_{R}^{}{e^{ -a^{2} \lambda_{1}^{2}}} \cdot e^{i \lambda_{1} x} d \lambda_{1}\right)\left(\frac{1}{2 \pi} \int_{R} e^{-a^{2} \lambda_{2}^{2} t} \cdot e^{i \lambda_{2} y} d \lambda_{2}\right) \\
& =\left(\frac{1}{2 a \sqrt{\pi t}} \cdot \exp \left\{-\frac{x^{2}}{4 a^{2} t}\right\}\right) \cdot\left(\frac{1}{2 a \sqrt{\pi t}} \exp \left\{-\frac{y^{2}}{4 a^{2} t}\right\}\right) \\
& =\frac{1}{4 a^{2} \pi t} \exp \left\{-\frac{x^{2}+y^{2}}{4 a^{2} t}\right\}
\end{aligned}
$$
于是 
$$u(x, y, t)=\frac{1}{4 a^{2} \pi t} \iint_{R^{2}} u(\xi, \eta) \exp \left\{-\frac{(x-\xi)^{2}+(y-\eta)^{2}}{4 a^{2} t}\right\} d \xi d \eta $$
由积分的绝对值不等式
有 
$$ |u(x, y, t)| \leqslant \frac{1}{4 a^{2} \pi t} \iint_{R^{2}}|u(\xi, \eta)| \exp \left\{-\frac{(x-\xi)^{2}+(y-\eta)^{2}}{4 a^{2} t}\right\} d \xi d \eta $$
$$
\leqslant \frac{1}{4 a^{2} \pi t} \iint_{R^{2}}|\varphi(\xi, \eta)| d \xi d \eta
$$
由于 $ \varphi(x, y) $ 为 $R^{2}$ 上的有界连续函数,
则 $$ |u(x, y, t)| \leqslant \frac{C}{t} \rightarrow 0 $$
其中 $C$ 为一个仅与 $a$ 及
$$\|R(\xi, \eta)\|_{L^{1}\left(R^{2}\right)}=\int_{-\infty}^{\infty}\int_{-\infty}^{\infty}|\varphi(\xi, \eta)| d \xi d \eta $$
有关的正常数.




\end{solution}
\end{questions}